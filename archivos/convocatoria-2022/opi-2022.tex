\documentclass{article}
\usepackage[utf8]{inputenc}

\title{Convocatoria a la Olimpiada Peruana de Informática
}
\author{Federación Olímpica Peruana de Informática (FOPI)}
%\date{Marzo 2021}
\date{\today}

\usepackage{natbib}
\usepackage{graphicx}
\usepackage{url}
\usepackage{xcolor}

\usepackage{geometry}
%\geometry{legalpaper}

\usepackage{hyperref}
\hypersetup{
    colorlinks=true,
    linkcolor=blue,
    filecolor=magenta,      
    urlcolor=blue,
}

\begin{document}

\maketitle

\section{Introducción}

La Federación Olímpica Peruana de Informática (FOPI) A.C. convoca a todos los alumnos que radiquen en Perú y estén inscritos en alguna institución educativa de nivel preuniversitario, a participar en la Olimpiada Peruana de Informática (OPI-2022). La misma tiene por objetivo seleccionar a los alumnos que conformen la delegación que representará al Perú en la Competencia Iberoamericana de Informática y Computación (CIIC-2022) y en la Olimpiada Internacional de Informática (IOI-2022).
En esta edición, la OPI tendrá dos etapas: FASE 1 y FASE 2.

Dudas serán resueltas por email 
({\color{blue}{olimpiadaperuanainformatica@gmail.com})}
%contacto@opi.org.pe
o por la página de facebook \url{https://www.facebook.com/InformaticaPe/}.


\section{Requisitos de participación en la OPI}

\begin{itemize}
    %\item Estar inscrito en una institución educativa peruana.
    \item Tener como máximo 20 años de edad al día 1 de Julio de 2022.
    \item Estar inscrito en alguna institución educativa peruana (colegio) en el 2022 o haber egresado el 2021.
    %\item 
    %En concordancia con el reglamento de la CIIC, en esta edición 
    %También podrán participar alumnos egresados el 2020 y que hayan estado inscritos en alguna academia preuniversitaria durante el 2021
\end{itemize}


%\begin{figure}[h!]
%\centering
%\includegraphics[scale=1.7]{universe}
%\caption{The Universe}
%\label{fig:universe}
%\end{figure}

\section{FASE 1 y FASE 2}

En esta edición, la OPI contará con dos fases. La primera de ellas consiste en captar los mejores talentos mediante una fase virtual. En esta primera etapa se puede tener un buen desempeño con conocimientos básicos de programación logrando clasificar a la siguiente etapa. La segunda etapa será presencial en la cual se escogerá a los representantes peruanos para las olimpiadas internacionales.
En caso de ser seleccionado, el alumno será responsable de transportarse a alguna de las sedes establecidas para rendir el examen.
 

Ambas fases serán realizadas a través de la plataforma OmegaUp (\url{https://omegaup.com/}).
    Es responsabilidad del participante familiarizarse con la plataforma previamente al concurso (\url{https://goo.gl/EaF2SX}).
      Los problemas serán propuestos por el Comité Científico de FOPI, el mismo que establecerá los casos oficiales de evaluación para cada problema
    
Para ambas fases, los problemas son de alguno de los siguientes tópicos, no excluyentes entre sí:
Algoritmos Constructivos (Ad-hoc);
Algoritmos de fuerza bruta (Backtracking, Búsqueda completa);
Algoritmos Voraces (Greedy);
Matemática Discreta,
Geometría, Aritmética,
Programación Dinámica;
Manejo de Cadenas (Strings);
Estructuras de datos;
Algoritmos en Grafos.

En esta oportunidad tendremos dos tipos de problemas, los cuales pasamos a describir.

\subsection*{Problemas de repuesta única}

Son problemas matemáticos y de lógica donde la programación es opcional.
Estos problemas son del mismo estilo del servidor Project Euler (\url{https://projecteuler.net}).
En ellos, se puede calcular la solución con lápiz y papel en un tiempo razonable, sin embargo, se lograría resolverlo mucho más rápido si se tienen conocimientos de programación.

\subsection*{Problemas de programación competitiva}

Son problemas clásicos de competiciones de informática.
(ver por ejemplo \url{https://omegaup.com/arena/CPIO-2021}). 
 Para enviar una solución, los participantes deberán enviar su código al juez en línea en los lenguajes de programación permitidos.
El juez proporciona uno de los siguientes veredictos para cada solución enviada:
Aceptado,
Parcialmente aceptado,
Respuesta incorrecta,
Tiempo límite excedido,
Memoria límite excedida,
Error en tiempo de ejecución,
Error de compilación,
Función restringida.

\subsection*{FASE 1}

\begin{itemize}

 \item La fecha de FASE 1 será el día {\color{red} 13 de Noviembre de 2021 a las 14h} . 
 
  \item Esta fase será {\color{red}{virtual}}.

  \item Serán de 2 a 3 problemas de tipo respuesta única  y 3 de programación competitiva, para resolver en 4 horas.
  
   \item Se podrá usar C++ o Python
   
   \item

Cada problema vale 100 puntos.
Los problemas de programación competitiva
podrán repartir estos puntos en puntajes parciales, los problemas de respuesta única solo tienen puntaje total o nulo.

 \item 
Existen 2 criterios para la asignación de puestos, en orden de prioridad: por puntaje obtenido, por penalidad de tiempo en caso de empate.
Es decir, la penalidad no quita puntos, se utiliza sólo en caso de empate.


\item Clasifican a la FASE 2 los primeros 30 alumnos con puntaje mayor a cero


\end{itemize}

\subsection*{FASE 2}

%Concurso
\begin{itemize}
    \item La fecha de la FASE 2 será el día {\color{red} 01 de Mayo de 2022 a las 14h} . 
    
    \item Esta fase será {\color{red}presencial}. Contaremos con 2 sedes en las ciudades de Lima y Arequipa.
    
   \item Serán de 4 a 5 problemas de programación competitiva para resolver en 5 horas.
   
 \item Se podrá usar solo C++.
   

\item

Cada problema equivale a un total de 100 puntos, que pueden estar repartidos en puntajes parciales.

 \item 
Existen 2 criterios para la asignación de puestos, en orden de prioridad: por puntaje obtenido, por penalidad de tiempo en caso de empate.
Es decir, la penalidad no quita puntos, se utiliza sólo en caso de empate.


\end{itemize}

\section{Registro}

El registro para la FASE 1 es online accediendo al siguiente enlace
{\color{red} hasta el 5 de Noviembre de 2021}:

\begin{center}
\url{https://olimpiadadeinformatica.org.pe/}
\end{center}

\section{Delegaciones CIIC-2022 e IOI-2022}

La CIIC es un concurso a nivel iberoamericano para jóvenes que tengan gusto y facilidad por resolver problemas prácticos mediante la lógica y el uso de computadoras. Su objetivo es promover el aprendizaje de las ciencias computacionales en los estudiantes y servir como preparación para las delegaciones nacionales que asisten a la Olimpiada Internacional de Informática (IOI).
El 2022 será la primera edición en donde una delegación peruana participe en la IOI, la cual es la olimpiada mundial por excelencia para competencias de informática a nivel escolar.

\begin{itemize}
\item 
La fecha de CIIC-2022 será entre Junio y Julio del 2022.
El concurso se realiza de forma sincronizada en diferentes sedes en los países participantes.

\item Los 10 primeros puestos de la FASE 2 serán seleccionados para participar en CIIC-2022.


\item 
La IOI-2022 se realizará del 7 hasta el 14 de Agosto del 2022.
El concurso se realiza de forma presencial en  Indonesia desde. Más información en
\url{https://ioi2022.id/}.

\item Los 4 primeros puestos de la FASE 2 serán seleccionados para participar en IOI-2022.


%comentado por la pandemia
%\item 
%En el Perú contaremos con 2 sedes en las ciudades de Lima y Arequipa.

%\item 
%Las 2 sedes peruanas realizarán el concurso a las 3pm, hora local.


\end{itemize}


\section{Ediciones anteriores}
Entrena con las ediciones anteriores del concurso CPIO (equivalente a FASE 2): 

\begin{itemize}
\item 2021: \url{https://omegaup.com/arena/CPIO-2021}
\item 2020: \url{https://omegaup.com/arena/replica-cpio-2020/}
\item 2018: \url{https://omegaup.com/arena/CPIO2018/}
\item 2017: \url{https://omegaup.com/arena/CPIO2017/}

\end{itemize}



\vspace*{1.5cm}



\hspace*{\fill}
Federación Olímpica Peruana de Informática A.C.

%\bibliographystyle{plain}
%\bibliography{references}
\end{document}
