\documentclass{article}
\usepackage[utf8]{inputenc}

\title{Convocatoria a la 3era Competencia Peruana de Informática Online (CPIO-2020)
}
\author{Federación Olímpica Peruana de Informática (FOPI)}
\date{Febrero 2020}

\usepackage{natbib}
\usepackage{graphicx}
\usepackage{url}
\usepackage{xcolor}

\usepackage{geometry}
%\geometry{legalpaper}

\usepackage{hyperref}
\hypersetup{
    colorlinks=true,
    linkcolor=blue,
    filecolor=magenta,      
    urlcolor=blue,
}

\begin{document}

\maketitle

\section{Introducción}

La Federación Olímpica Peruana de Informática (FOPI) A.C. convoca a todos los alumnos que radiquen en Perú y estén inscritos en alguna institución educativa de nivel secundario (o ser recién egresados), a participar en la 3era Competencia Peruana de Informática Online (CPIO-2020). La misma tiene por objetivo seleccionar a los alumnos que conformen la delegación que representará al Perú en la Competencia Iberoamericana de Informática y Computación (CIIC-2020).

Dudas serán resueltas por email 
({\color{blue}{olimpiadaperuanainformatica@gmail.com})}
%contacto@opi.org.pe
o por la página de facebook \url{https://www.facebook.com/InformaticaPe/}.


%\begin{figure}[h!]
%\centering
%\includegraphics[scale=1.7]{universe}
%\caption{The Universe}
%\label{fig:universe}
%\end{figure}

\section{CPIO-2020}

El CPIO-2020 es un concurso en línea cuyo objetivo
es seleccionar a los estudiantes que formarán parte de la delegación peruana a la CIIC-2020.
El registro es online accediendo al siguiente enlace
{\color{red} hasta el 21 de Abril del 2020}:

\begin{center}
\url{https://olimpiadadeinformatica.org.pe/}
\end{center}

\subsection{Requisitos}

\begin{itemize}
    \item Ser Peruano.
    \item Estar cursando del primero al quinto año de educación secundaria durante el año 2020 o haber cursado el quinto año de secundaria durante el 2019  y no haber iniciado la universidad.
    \item Tener como máximo 20 años de edad al día 30 de Mayo de 2020, día en que se realiza la CIIC.
\end{itemize}

%Registro
%El estudiante debe cumplir los siguientes requisitos:
%Ser Peruano.
%Estar cursando del primero al quinto año de educación secundaria durante el año 2020 o haber cursado el quinto año de secundaria durante el 2019  y no haber iniciado la universidad
%Tener como máximo 20 años de edad el día 30 de Mayo que se realiza la CIIC
%Registrarse en http://opi.org.pe hasta el 21 de Abril %del 2020.

\subsection{Sobre el concurso}

%Concurso
\begin{itemize}
    \item La fecha del concurso será el día {\color{red} Sábado 25 de Abril de 2020 a las 14h} . 

    
      \item Será realizado en línea a través de la plataforma OmegaUp (\url{https://omegaup.com/}).
    Es responsabilidad del participante familiarizarse con la plataforma previamente al concurso (\url{https://goo.gl/EaF2SX})
    
   \item Serán de 3 a 5 problemas para resolver en 5 horas.
   Los problemas serán propuestos por el Comité Científico del CPIO-2020, el mismo que establecerá los casos oficiales de evaluación para cada problema
   
    \item Los problemas podrán ser de alguno de los siguientes tópicos, no excluyentes entre sí:
Algoritmos Constructivos (Ad-hoc);
Algoritmos de fuerza bruta;
Algoritmos Voraces (Greedy);
Matemática Discreta,
Geometría, Aritmética y Álgebra;
Manejo de Cadenas (Strings);
Algoritmos en Grafos;
Programación Dinámica.

 \item 
 Para enviar una solución, los participantes deberán enviar su código al juez en línea en los lenguajes de programación C\texttt{++}  o Java.
El juez proporciona uno de los siguientes veredictos para cada solución enviada:
Aceptado,
Parcialmente aceptado,
Respuesta incorrecta,
Tiempo límite excedido,
Memoria límite excedida,
Error en tiempo de ejecución,
Error de compilación,
Función restringida.

\item

Cada problema equivale a un total de 100 puntos, que pueden estar repartidos en puntajes parciales.

 \item 
Existen 2 criterios para la asignación de puestos, en orden de prioridad: por puntaje obtenido, por penalidad de tiempo en caso de empate.
%Cada problema equivale a un total de 100 puntos, que pueden estar repartido en puntajes parciales.
Es decir, la penalidad no quita puntos, se utiliza sólo en caso de empate.

 \item 
Los 10 primeros estudiantes con un puntaje mayor a cero serán invitados a formar parte de la selección peruana en el CIIC-2020.

 \item 
En caso de ser seleccionado, el alumno será responsable de transportarse a alguna de las sedes establecidas para rendir el examen.
 
\end{itemize}

\section{CIIC-2020}

La CIIC es un concurso a nivel iberoamericano para jóvenes que tengan gusto y facilidad por resolver problemas prácticos mediante la lógica y el uso de computadoras. Su objetivo es promover el aprendizaje de las ciencias computacionales en los estudiantes y servir como preparación para las delegaciones nacionales que asisten a la Olimpiada Internacional de Informática (IOI).

\begin{itemize}
\item 
La fecha del concurso será el {\color{red} Sábado 30 de Mayo del 2020}.

\item 

El concurso se realiza de forma sincronizada en diferentes sedes en los países participantes.

\item 
En el Perú contaremos con 2 sedes en las ciudades de Lima y Arequipa.

\item 
Las 2 sedes peruanas realizarán el concurso a las 3pm, hora local.

\end{itemize}
\vspace*{1.5cm}

\hspace*{\fill}
Federación Olímpica Peruana de Informática A.C.

%\bibliographystyle{plain}
%\bibliography{references}
\end{document}
